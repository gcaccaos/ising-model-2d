\section{Manual do usuário}

Os códigos criados para obter os gráficos deste projeto estão no arquivo de notebook para o Jupyter, ``GabrielFariasCaccaos{\_}Projeto4.ipynb''. Para executá-lo, é necessária a versão 3.6.5 (ou superior) de Python e os módulos matplotlib, numpy, numba e ipywidgets (o qual é parte do IPython). \cite{scipy} Além disso, é necessário que o arquivo ``ising2d.py'' esteja no mesmo diretório, uma vez que contém as funções criadas para as simulações e suas ilustrações.

A função ``ising{\_}step'' efetua um passo na evolução de uma configuração de spins atraves do algoritmo de Metropolis para os valores de $T$, $J$ e $h$ fornecidos. O passo de Monte Carlo do algoritmo de Metropolis, por sua vez, é definido na função "ising update". Para obter a solução estacionária dos valores esperados da magnetização por sítio e da energia por sítio, é necessário escrever um script, em uma nova célula do notebook, consistindo de um loop de N passos com a função ``ising step'' aplicada iterativamente sobre as configurações de spins. Assim, é possível investigar a evolução desses valores esperados, a cada passo de Monte Carlo, através das funções ``magnetization{\_}per{\_}site'' e ``energy{\_}per{\_}site'' respectivamente. Claramente, o número de passos necessários para atingir o estado estacionário depende da largura do grid e da temperatura para uma dada configuração inicial. Por exemplo, para uma situação de baixa temperatura, mantendo-se a largura do grid, espera-se que uma configuração uniforme atinja um estado estacionário mais rapidamente do que uma configuração inicial aleatória. Portanto, para cada configuração inicial simulada com determinados valores de $L$, $J$ e $h$, há um número típico de passos que fazem parte de um regime transitório e que devem ser descartados na análise das grandezas médias do sistema.

Assim, um usuário pode manipular as variáveis $L$, $J$, $h$, $Nsteps$ e $T$, além da própria configuração inicial (primeiro índice do vetor ``steps''), a fim de variar os resultados das células do notebook.